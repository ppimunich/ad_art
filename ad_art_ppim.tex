\documentclass{article}
\usepackage[utf8]{inputenc}
\usepackage{graphicx}

\title{Anggaran Dasar dan Anggaran Rumah Tangga Perhimpunan Pelajar Indonesia di Munich}
\date{}

\newcommand{\sect}[1]{\addcontentsline{toc}{section}{#1}\section*{#1}}
\newcommand{\subsect}[1]{\addcontentsline{toc}{subsection}{#1}\subsection*{#1}}
\newcommand{\subsubsect}[1]{\addcontentsline{toc}{subsubsection}{#1}\subsubsection*{#1}}
\renewcommand{\contentsname}{Daftar Isi}

\begin{document}

\maketitle

\newpage

\tableofcontents

\newpage

\sect{Anggaran Dasar PPI Munich}

\subsect{Preambule}

Perhimpunan Pelajar Indonesia (PPI) Munich dibentuk dalam rangka mempererat hubungan persaudaraan antara para pelajar Indonesia yang sedang melanjutkan studi di München.

\subsect{Bab 1. Tata Organisasi}

\subsubsect{Pasal 1. Nama, Waktu dan Kedudukan}

\begin{enumerate}
    \item {Nama \\
    Nama organisasi adalah Perhimpunan Pelajar Indonesia di Munich atau disingkat PPI Munich. Diterjemahkan ke dalam bahasa Jerman menjadi \textit{Vereinigung Indonesischer Studenten in Deutschland niederlassung München}. Diterjemahkan ke dalam bahasa Inggris menjadi \textit{Indonesian Student Association in Germany branch Munich}.
    }
    \item{Waktu \\
    PPI Munich didirikan di Munich pada tahun 2010.
    }
    \item{Kedudukan \\
    PPI Munich berkedudukan di kota Munich, Bavaria, Jerman dan merupakan anggota dari Perhimpunan Pelajar Indonesia Jerman (PPI Jerman).
    }
\end{enumerate}

\subsubsect{Pasal 2. Asas dan Dasar}

PPI Munich berasaskan Pancasila dan berdasarkan UUD 1945. PPI Munich mengacu kepada Anggaran Dasar PPI Jerman.

\subsubsect{Pasal 3. Sifat dan Bentuk}

PPI Munich adalah organisasi yang bersifat politis non partais, kekeluargaan, dan mandiri dalam menentukan arah kebijakannya. 

\subsubsect{Pasal 4. Lambang}

\begin{enumerate}
    \item {
    Lambang PPI Munich berbentuk tulisan PPI Munich yang menyerupai tuts piano dan berwarna biru.
    \begin{center}
        \includegraphics[height=3cm,keepaspectratio]{image/logoPPI.png}
    \end{center}
    }
    \item {
    Makna lambang PPI Munich:
    \begin{itemize}
        \item {
        Tuts piano yang melambangkan keharmonisan,
        }
        \item {
        Warna biru sebagai warna dari provinsi Bavaria
        }
    \end{itemize}
    }
\end{enumerate}

\subsect{Bab 2. Tujuan}

\subsubsect{Pasal 5. Tujuan}

Pendirian PPI Munich memiliki maksud dan tujuan sebagai berikut:

\begin{enumerate}
    \item {
        Menjadi wadah komunikasi dan informasi bagi seluruh anggotanya,
    }
    \item {
        Memupuk persatuan dan kesatuan antar anggotanya,
    }
    \item {
        Mempererat hubungan persaudaraan dengan PPI yang ada di kota lain baik skala regional maupun internasional,
    }
    \item {
        Membina hubungan baik dengan masyarakat dan pelajar setempat.
    }
\end{enumerate}

\subsect{Bab 3. Keanggotaan}

\subsubsect{Pasal 6. Jenis dan Syarat Keanggotaan}

\begin{enumerate}
    \item {
        Anggota PPI Munich terdiri dari anggota penuh, anggota luar biasa, dan anggota kehormatan dengan kriteria sebagai berikut:
        \begin{enumerate}
            \item {
                Keanggotaan penuh PPI Munich bersifat terbuka bagi Warga Negara Indonesia pada salah satu lembaga perguruan tinggi dan/atau institusi pendidikan atau pengajaran yang setara dan bertempat tinggal di Munich,
            }
            \item {
                Anggota luar biasa adalah mereka dari segala kebangsaan yang menaruh simpati terhadap PPI Munich, tetapi tidak memenuhi salah satu persyaratan AD pasal 6.1.a,
            }
            \item {
                Anggota kehormatan adalah mereka yang telah berjasa kepada PPI Munich dan/atau PPI Jerman.
            }
        \end{enumerate}
    }
    \item {
        Hal-hal selanjutnya mengenai keanggotaan diatur dalam Anggaran Rumah Tangga.
    }
\end{enumerate}

\subsubsect {Pasal 7. Hak Anggota}

\begin{enumerate}
    \item {
        Anggota penuh berhak:
        \begin{enumerate}
            \item {
                Memperoleh informasi dari PPI Munich,
            }
            \item {
                Mengikuti kegiatan-kegiatan PPI Munich,
            }
            \item {
                Menggunakan fasilitas yang dimiliki oleh PPI Munich sesuai dengan aturan yang berlaku,
            }
            \item {
                Hadir, menyatakan pendapat dan suara secara bebas dan bertanggung jawab dalam pengambilan keputusan dalam Musyawarah Anggota,
            }
            \item {
                Memegang jabatan dalam perhimpunan,
            } 
            \item {
                Mengusulkan calon anggota kehormatan.
            }
        \end{enumerate}
    }
    \item {
        Anggota luar biasa berhak:
        \begin{enumerate}
            \item {
                Hadir dan menyatakan pendapat secara bebas dan bertanggung jawab dalam rapat perhimpunan,
            }
            \item {
                Mengikuti kegiatan-kegiatan yang bersifat umum,
            }
            \item {
                Mengusulkan calon anggota kehormatan.
            }
        \end{enumerate}
    }
    \item {
        Anggota kehormatan berhak:
        \begin{enumerate}
            \item {
                Hadir dan menyatakan pendapat secara bebas dan bertanggung jawab dalam rapat perhimpunan,
            }
            \item {
                Mengikuti kegiatan-kegiatan yang bersifat umum.
            }
        \end{enumerate}
    }
\end{enumerate}

\subsubsect {Pasal 8. Kewajiban Anggota}

\begin{enumerate}
    \item {
        Anggota penuh wajib:
        \begin{enumerate}
            \item {
                Menjunjung tinggi dan menaati AD/ART PPI Munich,
            }
            \item {
                Melaksanakan hasil-hasil rapat perhimpunan termasuk Musyawarah Anggota.
            }
        \end{enumerate}
    }
    \item {
        Anggota luar biasa wajib:
        \begin{enumerate}
            \item {
                Menjunjung tinggi dan menaati AD/ART PPI Munich,
            }
            \item {
                Melaksanakan hasil-hasil rapat perhimpunan termasuk Musyawarah Anggota.
            }
        \end{enumerate}
    }
    \item {
        Anggota kehormatan wajib:
        \begin{enumerate}
            \item {
                Menjunjung tinggi dan menaati AD/ART PPI Munich,
            }
            \item {
                Melaksanakan hasil-hasil rapat perhimpunan termasuk Musyawarah Anggota.
            }
        \end{enumerate}
    }
\end{enumerate}

\subsect {Bab 4. Badan Kelengkapan Perhimpunan}

\subsubsect {Pasal 9. Badan Kelengkapan Perhimpunan}

\begin{enumerate}
    \item {
        Musyawarah Anggota (MA) yang merupakan pemegang kekuasaan tertinggi dalam organisasi PPI Munich.
    }
    \item {
        Musyawarah Anggota Luar Biasa (MA-LB) adalah musyawarah anggota yang dilaksanakan dalam keadaan darurat.
    }
    \item {
        Pengurus Cabang (PC) yang merupakan Lembaga yang menjalankan peran eksekutif dan bertanggung jawab atas keberlangsungan organisasi PPI Munich serta bertanggung jawab kepada MA.
    }
    \item {
        Dewan Pengawas Organisasi (DPO) yang merupakan Lembaga yang menjalankan fungsi kontrol bagi PPI Munich.
    }
    \item {
        Hal - hal selanjutnya mengenai kelengkapan yang telah disebutkan di dalam ayat 1, 2, 3, dan 4 diatur dalam Anggaran Rumah Tangga PPI Munich
    }
\end{enumerate}

\subsect {Bab 5. Fungsi dan Tugas Badan Kelengkapan Perhimpunan}

\subsubsect {Pasal 10. Musyawarah Anggota (MA) dan Musyawarah Anggota Luar Biasa (MA-LB)}

\begin{enumerate}
    \item {
        Musyawarah Anggota atau MA
        \begin{enumerate}
            \item {
                memiliki kekuatan hukum tertinggi dalam perhimpunan,
            }
            \item {
                bersidang sekurang-kurangnya satu kali dalam masa kepengurusan,
            }
            \item {
                berfungsi untuk memilih ketua pengurus pusat dan ketua DPO,
            }
            \item {
                memiliki wewenang untuk mengubah AD/ART, jika dianggap perlu,
            }
            \item {
                memiliki wewenang untuk menerima Laporan Pertanggungjawaban Pengurus dan DPO,            
            }
            \item {
                berhak menerima atau menolak Laporan Pertanggungjawaban Pengurus dan DPO,
            }
            \item {
                mengesahkan garis-garis besar Program Kerja PPI Munich,
            }
            \item {
                mengesahkan anggota kehormatan,
            }
            \item {
                peninjauan dan pencabutan keputusan-keputusan MA hanya dapat dilakukan di MA,
            }
            \item {
                keputusan MA mengikat setiap anggota dan berlaku dalam segala badan kelengkapan perhimpunan.
            }
        \end{enumerate}
    }
    \item {
        Musyawarah Anggota Luar Biasa atau MA-LB
        \begin{enumerate}
            \item {
                memiliki kekuatan hukum sama seperti MA,
            }
            \item {
                dilaksanakan dalam keadaan darurat,
            }
            \item {
                hanya dapat dijalankan jika diusulkan oleh anggota dan didukung oleh sekurang-kurangnya 50\% + 1 dari jumlah anggota dalam pengurus cabang.
            }
        \end{enumerate}
    }
\end{enumerate}

\subsubsect{Pasal 11. Pengurus Cabang (PC)}

\begin{enumerate}
    \item {
        Pengurus cabang (PC)
        \begin{enumerate}
            \item {
                Melaksanakan keputusan-keputusan MA,
            }
            \item {
                Melakasanakan keputusan-keputusan Sidang Perwakilan Anggota PPI Jerman. 
            }
        \end{enumerate}
    }
\end{enumerate}

\subsubsect {Pasal 12. Dewan Pengawas Organisasi}

\begin{enumerate}
    \item {
        Dewan Pengawas Organisasi
        \begin{enumerate}
            \item {
                Melaksanakan keputusan-keputusan MA,
            }
            \item {
                Melakasanakan keputusan-keputusan Sidang Perwakilan Anggota PPI Jerman. 
            }
        \end{enumerate}
    }
\end{enumerate}

\subsect{Bab 6. Pengambilan Keputusan}

\subsubsect {Pasal 13. Keputusan Musyawarah}

\begin{enumerate}
    \item {
        Setiap keputusan dalam perhimpunan diambil secara musyawarah dan mufakat.
    }
    \item {
        Musyawarah dilaksanakan berdasarkan gotong-royong dengan sikap saling memberi dan menerima dalam suasana kekeluargaan dan toleransi antara segenap peserta musyawarah.
    }
    \item {
        Musyawarah bertujuan untuk mencari kesatuan pendapat atau kesadaran dan rasa tanggung jawab di antara segenap peserta musyawarah.
    }
    \item {
        Apabila musyawarah tidak menghasilkan mufakat, maka diadakan pemungutan suara. Keputusan diambil dengan suara terbanyak.
    }
    \item {
        Setiap Musyawarah Anggota diatur berdasarkan tata tertib Musyawarah Anggota yang dibuat oleh Organizing Committee Musyawarah Anggota dan disahkan di Musyawarah Anggota.
    }
\end{enumerate}

\subsect{Bab 7. Keuangan}

\subsubsect {Pasal 14. Keuangan}

\begin{enumerate}
    \item {
        Keuangan perhimpunan didapatkan dari: 
        \begin{enumerate}
            \item {
                saldo kas kepengurusan sebelumnya,
            }
            \item {
                hasil-hasil usaha yang halal, sah, dan tidak bertentangan dengan AD/ART,
            }
            \item {
                Sumbangan-sumbangan yang tidak mengikat terhadap PPI Munich, salah satunya sumbangan anggota.
            }
        \end{enumerate}
    }
    \item {
        Keuangan dipergunakan untuk membiayai keperluan di perhimpunan dalam mewujudkan visi dan misi perhimpunan.
    }
    \item {
        Pengawas keuangan diatur lebih lanjut dalam Anggaran Rumah Tangga.
    }
\end{enumerate}

\subsect{Bab 8. Aturan Tambahan}

\subsubsect {Pasal 15. Perubahan AD/ART}

\begin{enumerate}
    \item {
        Kehendak perubahan AD/ART PPI Munich dapat diajukan oleh seluruh anggota penuh PPI Munich.
    }
    \item {
        Kehendak perubahan AD/ART PPI Munich tersebut harus disepakati secara tertulis Pengurus Cabang.
    }
    \item {
        Perubahan AD/ART harus dilaksanakan dalam Musyawarah Anggota dengan persetujuan dari sekurang-kurangnya 50\% + 1 dari jumlah anggota perhimpunan yang hadir di Musyawarah Anggota.    
    }
\end{enumerate}

\subsubsect{Pasal 16. Pembubaran Perhimpunan}

\begin{enumerate}
    \item {
        Pembubaran perhimpunan dapat dilakukan dalam Musyawarah Anggota yang acara pembubarannya  harus  diberitahukan  kepada  segenap  anggota  perhimpunan setahun sebelumnya.
    }
    \item {
        Pembubaran harus disetujui oleh seluruh anggota PPI Munich.
    }
    \item {
        Apabila PPI Munich dibubarkan, maka kepemilikan inventaris PPI Munich akan ditetapkan dalam Musyawarah Anggota.
    }
\end{enumerate}

\subsubsect{Pasal 17. Hal Lain-Lain}

Hal lain - lain yang tidak tercantum dalam Anggaran Dasar, diatur dalam Anggaran Rumah Tangga dan peraturan-peraturan yang diputuskan oleh Musyawarah Anggota.

\newpage

\sect{Anggaran Rumah Tangga PPI Munich}

\subsubsect{Pasal 1. Keanggotaan}

\begin{enumerate}
    \item {
    Anggota penuh PPI Munich yang diakui keabsahan status keanggotaanya adalah mereka yang terdaftar pada sistem sensus PPI Munich.
    }
    \item {
    Anggota luar biasa PPI Munich yang diakui keabsahan status keanggotaannya adalah mereka yang terdaftar pada sistem sensus PPI Munich dan telah memberikan sumbangan anggota dengan nominal minimal yang ditetapkan oleh pengurus cabang pada periode tersebut.
    }
\end{enumerate}

\subsubsect{Pasal 2. Pencabutan Keanggotaan}

\begin{enumerate}
    \item {
        Seseorang kehilangan status keanggotaannya di PPI Munich apabila yang bersangkutan
        \begin{enumerate}
            \item {
                Mengundurkan diri secara tertulis kepada Pengurus Cabang PPI Munich,
            }
            \item {
                Dicabut oleh Pengurus Cabang PPI Munich,
            }
            \item {
                Melakukan tindak kriminal dan/atau tindak yang bertentangan dengan asas atau dasar PPI Munich.
            }
        \end{enumerate}
    }
    \item {
        Seseorang kehilangan status keanggotaannya di PPI Munich otomatis apabila PPI Munich dibubarkan.
    }
\end{enumerate}

\subsubsect{Pasal 3. Musyawarah Anggota dan Musyawarah Anggota Luar Biasa}

\begin{enumerate}
    \item {
        Musyawarah Anggota atau MA
        \begin{enumerate}
            \item {
                MA dapat diadakan apabila diusulkan oleh sekurang – kurangnya seperempat ($\frac{1}{4}$) jumlah anggota penuh PPI Munich atau Pengurus Cabang PPI Munich dengan alasan tertulis dan disetujui oleh ketua PPI Munich. 
            }
            \item {
                Pengusul MA diwajibkan menghadiri MA yang diusulkan oleh yang bersangkutan.
            }
            \item {
                Musyawarah dinyatakan sah apabila dihadiri oleh 50\% + 1 anggota penuh PPI Munich yang telah menyatakan konfirmasi kehadiran.
            }
            \item {
                Apabila kuorum tidak terpenuhi, maka MA ditunda sampai waktu yang ditentukan oleh Pengurus Cabang PPI Munich. 
            }
            \item {
                MA, yang diadakan karena kuorum pada MA sebelumnya tidak terpenuhi (sesuai pasal 3 ayat 1.c ART PPI Munich), tidak memiliki kuorum.
            }
            \item {
                MA dikoordinir oleh suatu Organizing Committee yang ditunjuk oleh PPI Munich dan dipimpin oleh pimpinan sidang yang disepakati dalam MA.
            }
            \item {
                MA diumumkan sekurang-kurangnya empat (4) minggu sebelum MA diselenggarakan.
            }
            \item {
                Notulen MA dibuat oleh panitia penyelenggara MA dan disahkan dalam MA tersebut.
            }
            \item {
                Anggota sidang yang meninggalkan ruang/tempat sidang itu tanpa seijin presidium akan kehilangan hak suara pada sidang tersebut.
            }
            \item {
                Hal-hal mengenai MA yang tidak diatur dalam Anggaran Rumah Tangga ini diatur lebih lanjut oleh peraturan yang ditetapkan oleh presidium musyawarah anggota selama tidak bertentangan dengan Anggaran Rumah Tangga ini.
            }
        \end{enumerate}
    }
    \item {
        Musyawarah Anggota dapat diselenggarakan dengan agenda sebagai berikut tetapi tidak terbatas pada:
        \begin{enumerate}
            \item {
                Pengangkatan dan penurunan ketua PPI Munich serta pembubaran Pengurus Cabang,
            }
            \item {
                Musyawarah kerja kepengurusan,
            }
            \item {
                Evaluasi kepengurusan melalui laporan pertanggungjawaban dalam masa atau akhir kepengurusan,
            }
            \item {
                Pengesahan perubahan AD dan/atau ART PPI Munich,
            }
            \item {
                Pemberian sanksi kepada anggota PPI Munich,
            }
            \item {
                Pembahasan penting lain yang disetujui oleh Pengurus Cabang.
            }
        \end{enumerate}
    }
    \item {
        Musyawarah Anggota Luar Biasa atau MA-LB
        \begin{enumerate}
            \item {
                Musyawarah Anggota Luar Biasa dapat dilaksanakan apabila disetujui oleh anggota penuh PPI Munich yang bukan merupakan pengurus PPI Munich pada periode tersebut sekurang-kurangnya sejumlah pengurus PPI Munich pada periode itu ditambah satu atau oleh DPO.
            }
            \item {
                Musyawarah dinyatakan sah apabila dihadiri oleh dua per tiga ($\frac{2}{3}$) ditambah 1 anggota penuh PPI Munich yang telah menyatakan konfirmasi kehadiran.
            }
            \item {
                Ketentuan ART lain mengenai MA-LB diatur oleh pasal 3 ayat 1 Anggaran Rumah Tangga.
            }
        \end{enumerate}
    }
\end{enumerate}

\subsubsect{Pasal 4. Pengurus Cabang}

\begin{enumerate}
    \item {
        Syarat Pengurus Cabang \\
        Merupakan anggota penuh PPI Munich selama periode kepengurusan cabang PPI Munich.
    }
    \item {
        Pembentukan 
        \begin{enumerate}
            \item {
                Ketua yang terpilih oleh MA atau MA-LB membentuk kepengurusan selambat-lambatnya 6 minggu setelah MA tersebut dilaksanakan.
            }
            \item {
                Apabila ketua terpilih tidak dapat melaksanakan ART pasal 4.2.a, maka akan diadakan MA untuk mengambil langkah-langkah yang diperlukan.
            }
            \item {
                Pengangkatan dan pembebastugasan pengurus cabang PPI Jerman merupakan hak prerogatif ketua.
            }
        \end{enumerate}
    }
    \item {
        Susunan 
        \begin{enumerate}
            \item {
                Pengurus cabang terdiri dari pengurus harian yang meliputi namun tidak terbatas pada ketua, wakil ketua, sekretaris jenderal, bendahara; dan pengurus bidang yang diangkat sesuai kebutuhan oleh ketua.
            }
            \item {
                Ketua mewakili perhimpunan ke luar dan ke dalam.
            }
            \item {
                Wakil ketua dapat mewakili ketua apabila ketua berhalangan.
            }
            \item {
                Apabila terjadi kekosongan jabatan ketua, maka jabatan tersebut akan diisi oleh hierarki jabatan setingkat di bawahnya dan ketua yang baru wajib menjalankan amanah MA sebelumnya sampai habis masa jabatan kepengurusan tersebut.
            }
        \end{enumerate}
    }
    \item {
        Laporan Pertanggungjawaban \\
        Pengurus cabang berkewajiban melaporkan dan mempertanggungjawabkan segala tindakannya dalam MA secara tertulis.
    }
    \item {
        Rapat \\
        Rapat pengurus cabang diadakan menurut keperluan.
    }
\end{enumerate}

\subsubsect{Pasal 5. Ketua} 

\begin{enumerate}
    \item {
        Kriteria calon ketua PPI Munich adalah sebagai berikut:
        \begin{enumerate}
            \item {
                Merupakan anggota penuh PPI Munich dan tidak sedang menjalani sanksi,
            }
            \item {
                Bukan merupakan pengurus dan/atau anggota aktif di salah satu partai politik,
            }
            \item {
                Bukan merupakan pengurus di organisasi PPI cabang lain selain PPI Munich,
            }
            \item {
                Bukan merupakan panitia pemilihan ketua PPI Munich,
            }
            \item {
                Memenuhi syarat dan kriteria tambahan yang diatur oleh panitia pemilihan umum ketua PPI Munich.
            }
        \end{enumerate}
    }
    \item {
        Ketua PPI Munich dapat diberhentikan apabila yang bersangkutan
        \begin{enumerate}
            \item {
                mengundurkan diri,
            }
            \item {
                melakukan pelanggaran hukum berupa pengkhianatan terhadap NKRI, korupsi, penyuapan, dan/atau tindak pidana berat lainnya sesuai dengan hukum perundang-undangan yang berlaku di Indonesia dan/atau di Jerman,
            }
            \item {
                tidak dapat memenuhi kewajibannya sebagai ketua PPI Munich,
            }
            \item {
                mangkat atau wafat.
            }
        \end{enumerate}
    }
    \item {
        Pengunduran diri ketua PPI Munich harus dilakukan secara tertulis dan diserahkan kepada Pengurus Cabang dan disetujui oleh MA.
    }
    \item {
        Ketua menjabat dalam periode satu tahun.
    }
    \item {
        Seorang anggota penuh hanya dapat menjabat sebagai ketua sebanyak maksimal dua (2) periode masa jabatan.
    }
\end{enumerate}

\subsubsect{Pasal 6. Dewan Pengawas Organisasi}

\begin{enumerate}
    \item {
        Syarat Dewan Pengawas Organisasi \\
        merupakan anggota PPI Munich selama periode kepengurusan PPI Munich dan pernah menjadi pengurus harian PPI Munich pada periode yang telah berlalu.
    }
    \item {
        Pembentukan
        \begin{enumerate}
            \item {
                Ketua Dewan Pengawas Organisasi adalah ketua sah yang ditunjuk melalui pemilihan dalam Musyawarah Anggota.
            }
            \item {
                Dewan Pengawas Organisasi terdiri atas minimal 1 orang.
            }
            \item {
                Dewan Pengawas Organisasi menjabat bersamaan dengan masa jabatan Pengurus Cabang pada periode tersebut.
            }
        \end{enumerate}
    }
    \item {
        Hak Dewan Pengawas Organisasi
        \begin{enumerate}
            \item {
                Meminta Laporan Pertanggungjawaban kepada Pengurus Cabang dengan waktu tenggat pemenuhan permintaan sesuai yang telah disepakati oleh Dewan Pengawas Organisasi dan Pengurus Cabang.
            }
            \item {
                Melakukan audit terhadap kinerja serta keuangan PPI Munich.
            }
            \item {
                Mengusulkan MA-LB
            }
        \end{enumerate}
    }
    \item {
        Kewajiban Dewan Pengawas Organisasi 
        \begin{enumerate}
            \item {
                Memberikan pertanggungjawaban minimal sekali setahun dalam Musyawarah Anggota.
            }
            \item {
                Mengadakan evaluasi terhadap kinerja PPI Munich minimal sekali dalam kurun waktu enam (6) bulan.
            }
            \item {
                Memberikan masukan dan saran terhadap PPI Munich apabila diminta oleh PPI Munich.
            }
            \item {
                Memberikan hasil evaluasi-evaluasi yang diadakan kepada PPI Munich apabila diminta.
            }
        \end{enumerate}
    }
\end{enumerate}

\subsubsect{Pasal 7. Hal Lain-Lain}

Hal lain - lain yang tidak tercantum di dalam Anggaran Rumah Tangga PPI Munich diatur oleh aturan - aturan yang dikeluarkan oleh Pengurus Cabang dan diketahui oleh anggota PPI Munich dan tidak boleh bertentangan dengan Anggaran Dasar dan Anggaran Rumah Tangga PPI Munich.

\vspace{1cm}

\begin{flushleft}
    \textbf{Amandemen AD/ART ini diajukan dan disahkan pada Musyawarah Anggota PPI Munich 2021 yang diselenggarakan pada tanggal 11 dan 12 Desember 2021 melalui media daring, dan berlaku sejak periode kepengurusan
    PPI Jerman 2021/2022.}
\end{flushleft}

\vspace{1cm}

\begin{minipage}{0.3\textwidth}
    \begin{center}
        Ketua PPI Munich 2020/2021 \\
        \vspace{2cm}
        Yudhistira A. Wibowo
    \end{center}
\end{minipage}
\begin{minipage}{0.3\textwidth}
    \begin{center}
        Ketua Musyawarah Anggota PPI Munich 2021 \\
        \vspace{2cm}
        Jeremialie K. Swadiryus
    \end{center}
\end{minipage}
\begin{minipage}{0.3\textwidth}
    \begin{center}
        Ketua Presidium Musyawarah Anggota PPI Munich 2021 \\
        \vspace{2cm}
        Jason R. Bradley
    \end{center}
\end{minipage}

\end{document}
